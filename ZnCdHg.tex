\documentclass[hyperref=unicode, presentation,10pt]{beamer}

\usepackage[absolute,overlay]{textpos}
\usepackage{array}
\usepackage{graphicx}
\usepackage{adjustbox}
\usepackage{mhchem}
\usepackage{chemfig}
\usepackage{caption}

%dělení slov
\usepackage{ragged2e}
\let\raggedright=\RaggedRight
%konec dělení slov

\addtobeamertemplate{frametitle}{
	\let\insertframetitle\insertsectionhead}{}
\addtobeamertemplate{frametitle}{
	\let\insertframesubtitle\insertsubsectionhead}{}

\makeatletter
\CheckCommand*\beamer@checkframetitle{\@ifnextchar\bgroup\beamer@inlineframetitle{}}
\renewcommand*\beamer@checkframetitle{\global\let\beamer@frametitle\relax\@ifnextchar\bgroup\beamer@inlineframetitle{}}
\makeatother
\setbeamercolor{section in toc}{fg=red}
\setbeamertemplate{section in toc shaded}[default][100]

\usepackage{fontspec}
\usepackage{unicode-math}

\usepackage{polyglossia}
\setdefaultlanguage{czech}

\def\uv#1{„#1“}

\mode<presentation>{\usetheme{default}}
 \usecolortheme{crane}

\setbeamertemplate{footline}[frame number]

\title[Crisis]
{C2062 -- Anorganická chemie II}

\subtitle{Zinek, kadmium, rtuť a kopernicium}
\author{Zdeněk Moravec, hugo@chemi.muni.cz \\ \adjincludegraphics[height=60mm]{img/IUPAC_PSP.jpg}}
\date{}

\begin{document}

\begin{frame}
	\titlepage
\end{frame}

\section{Úvod}
\frame{
	\frametitle{}
	\vfill
	\begin{tabular}{|c|l|l|l|}
	\hline
	 & \textit{Zinek} & \textit{Kadmium} & \textit{Rtuť} \\\hline
	 El. konfigurace & 3d$^{10}$ 4s$^{2}$ & 4d$^{10}$ 5s$^{2}$ & 4f$^{14}$ 5d$^{10}$ 6s$^{2}$ \\\hline
	 Teplota tání [$^\circ$C] & 420 & 321 & $-$39 \\\hline
	 Teplota varu [$^\circ$C]  & 907 & 767 & 357 \\\hline
	 Objeven & před 1000 př.n.l. & 1817 & před 1500 př.n.l. \\\hline
	 Vzhled & Stříbrošedý\footnote[frame]{Zdroj: \href{https://commons.wikimedia.org/wiki/File:Zinc_fragment_sublimed_and_1cm3_cube.jpg}{Alchemist-hp/Commons}} & stříbřitě modrošedý\footnote[frame]{Zdroj: \href{https://commons.wikimedia.org/wiki/File:Cadmium-crystal_bar.jpg}{Alchemist-hp/Commons}} & stříbrná kap.\footnote[frame]{Zdroj: \href{https://commons.wikimedia.org/wiki/File:Pouring_liquid_mercury_bionerd.jpg}{Bionerd/Commons}} \\
	 &  \begin{minipage}{.2\textwidth}
	 	\adjincludegraphics[width=\linewidth]{img/Zinc_fragment_sublimed_and_1cm3_cube.jpg}
	 \end{minipage}
	 	& \begin{minipage}{.2\textwidth}
	 		\adjincludegraphics[width=\linewidth]{img/Cadmium-crystal_bar.jpg}
	 	\end{minipage} & \begin{minipage}{.2\textwidth}
	 	\adjincludegraphics[width=\linewidth]{img/Pouring_liquid_mercury_bionerd.jpg}
 	\end{minipage} \\\hline
	\end{tabular}
	\vfill
}

\section{Kopernicium}
\frame{
	\frametitle{}
	\vfill
	\textbf{Kopernicium}
	\begin{columns}
		\begin{column}{.65\textwidth}
			\begin{itemize}
				\item Umělý prvek, protonové číslo 112, Cn.
				\item Poprvé byl připraven v roce 1991 v Darmstadtu.\footnote[frame]{\href{https://doi.org/10.1007/BF02769517}{The new element 112}}
				\item \ce{^{208}_{82}Pb + ^{70}_{30}Zn -> ^{278}_{112}Cn^* -> ^{277}_{112}Cn + ^1_0n}
				\item Nejstabilnějším izotopem je \ce{^{285}Cn} s poločasem rozpadu 30~s.\footnote[frame]{\href{https://dx.doi.org/10.1088/1674-1137/abddae}{The NUBASE2020 evaluation of nuclear physics properties}}
				\item \ce{^{285}_{112}Cn -> ^{281}_{110}Ds + $\alpha$}
				\item 19. února 2010 byl IUPACem schválen název Kopernicium, Cn, na počest polského astronoma a matematika Mikuláše Koperníka (1473--1543).\footnote[frame]{\href{https://www.sciencedaily.com/releases/2010/02/100224102231.htm}{Chemical element 112 is officially named 'Copernicium'}}
			\end{itemize}
		\end{column}

		\begin{column}{.4\textwidth}
			\begin{figure}
				\adjincludegraphics[width=\textwidth]{img/Nikolaus_Kopernikus.jpg}
				\caption*{Mikuláš Koperník.\footnote[frame]{Zdroj: \href{https://en.wikipedia.org/wiki/File:Nikolaus_Kopernikus.jpg}{District Museum in Toruń/Commons}}}
			\end{figure}
		\end{column}
	\end{columns}
	\vfill
}

\section{Chemické a fyzikální vlastnosti}
\subsection{Zinek}
\frame{
	\frametitle{}
	\vfill
	\textbf{Zinek}
	\begin{itemize}
		\item Diamagnetický kov, krystaluje v hexagonální soustavě.
		\item Dobrý vodič elektřiny.
		\item Má pět stabilních izotopů a 25 charakterizovaných radioizotopů.
	\end{itemize}
	\begin{center}
		\begin{tabular}{|l|r@{,}l|}
			\hline
			64 & 49 & 2 \\\hline
			66 & 27 & 7 \\\hline
			67 & 4 & 0 \\\hline
			68 & 18 & 5 \\\hline
			70 & 0 & 6 \\\hline
		\end{tabular}
	\end{center}
	\begin{itemize}
		\item Preferuje oxidační číslo II, ale je známo i několik sloučenin v oxidačním čísle I.
		\item Je reaktivnější než měď.
		\item Na vlhkém vzduchu rychle ztrácí lesk.
		\item Reaguje s kyslíkem, sírou a fosforem. Při zahřívání s halogeny.
		\item Z minerálních kyselin uvolňuje vodík:
		\item \ce{Zn + 2 HCl -> ZnCl2 + H2}
		\item Má redukční vlastnosti.
	\end{itemize}
	\vfill
}

\subsection{Kadmium}
\frame{
	\frametitle{}
	\textbf{Kadmium}
	\vfill
	\begin{itemize}
		\item Měkký kov, odolný vůči korozi.
		\item Je toxické.
		\item Přírodní kadmium se skládá ze šesti stabilních izotopů a dvou radioizotopů s velmi dlouhým poločasem rozpadu. Dále známe 38 radioizotopů.
	\end{itemize}
	\begin{center}
	\begin{tabular}{|l|r@{,}l|l|}
		\hline
		106 & 1 & 25 &  \\\hline
		108 & 0 & 89 & \\\hline
		110 & 12 & 49 & \\\hline
		111 & 12 & 80 & \\\hline
		112 & 24 & 13 & \\\hline
		113 & 12 & 22 & 7,7 $\times$ 10$^{15}$ let\\\hline
		114 & 28 & 73 & \\\hline
		116 & 7 & 49 & 3,1 $\times$ 10$^{19}$ let\\\hline
	\end{tabular}
	\end{center}
	\begin{itemize}
		\item Chemicky je podobné zinku, vytváří sloučeniny v oxidačním čísle II, velmi výjimečně i I.
		\item Na vzduchu hoří za vzniku amorfního \ce{CdO}.
		\item \ce{2 Cd + O2 -> 2 CdO}
	\end{itemize}
	\vfill
}

\frame{
	\frametitle{}
	\vfill
	\begin{itemize}
		\item Toxicita kadmia je dána tím, že kadmium vstupuje do metabolických drah zinku. Tím tyto dráhy narušuje.\footnote[frame]{\href{https://www.wikiskripta.eu/w/Kontaminace_kovy\#Kadmium}{Kontaminace kovy}}
		\item Otravu je možné potlačit podáváním zinku.
		\item Při inhalaci dochází primárně k poškození plic.
		\item Kadmium může také do těla vstupovat kůží.
		\item Velkým problémem při otravě kadmiem je dlouhý poločas jeho eliminace, takže může docházet k postupné akumulaci kadmia v organismu i při expozici nižším dávkám.
		\item Při projevu symptomů jsou následky otravy nevratné a dochází k postupnému zhoršování stavu.
		\item Kadmium může také podpořit rozvoj rakoviny plic a prostaty. Na druhou stranu, u některých nádorů mohou může kadmium působit jako supresivní látka.
	\end{itemize}
	\vfill
}

\subsection{Rtuť}
\frame{
	\frametitle{}
	\vfill
	\textbf{Rtuť}
	\begin{itemize}
		\item Kapalný, toxický kov s vysokou hustotou.
		\item Dobrý vodič elektřiny, ale špatně vede teplo.
		\item Má nejnižší teplotu varu i tání ze všech stabilních kovů.
		\item Má sedm stabilních izotopů a 43 radioizotopů.
	\end{itemize}
	\begin{center}
		\begin{tabular}{|l|r@{,}l|}
			\hline
			196 & 0 & 15 \\\hline
			198 & 10 & 04 \\\hline
			199 & 16 & 94 \\\hline
			200 & 23 & 14 \\\hline
			201 & 13 & 17 \\\hline
			202 & 29 & 74 \\\hline
			204 & 6 & 82 \\\hline
		\end{tabular}
	\end{center}
	\begin{itemize}
		\item Chemicky se mírně liší od zinku a kadmia.
		\item S kyselinami nereaguje, s výjimkou oxidujících -- dusičnou a koncentrovanou kyselinu sírovou.
		\item Kovy rozpouští za vzniku \textit{amalgámů}.
	\end{itemize}
	\vfill
}

\frame{
	\frametitle{}
	\vfill
	\begin{itemize}
		\item Rtuť je toxická ve všech formách, jako kov i jako anorganické a organokovové sloučeniny \ce{Hg^{2+}} a \ce{Hg$_2^{2+}$}.\footnote[frame]{\href{https://www.wikiskripta.eu/w/Intoxikace\_rtut\%C3\%AD\_a\_jej\%C3\%ADmi\_slou\%C4\%8Deninami}{Intoxikace rtutí a jejími sloučeninami}}
		\item K intoxikaci může dojít jak vlivem přírodních jevů (ze zemské kůry se uvolňují velká množství rtuti), tak vlivem průmyslové činnosti (těžba zlata, elektrolytické procesy, apod.).\footnote[frame]{\href{https://www.epa.gov/international-cooperation/mercury-emissions-global-context}{Mercury Emissions: The Global Context}}
		\item Kvůli vysoké těkavosti jsou často vdechovány páry rtuti, která pak prostupuje z plic do dalších orgánů (ledvin, CNS, červených krvinek).
		\item Vysoká mobilita rtuti v organismu je dána její rozpustností v tucích, což umožňuje transport přes buněčné membrány.
		\item Při chronické expozici dochází k poškozování CNS, které se projevuje třesavkou, emocionální nestabilitou a změnami chování. Dochází také k poškození ledvin a v případě těhotných žen i k poškození plodu.
	\end{itemize}
	\vfill
}

\frame{
	\frametitle{}
	\vfill
	\begin{itemize}
		\item Při otravě rtutí se využívají chelatační činidla, které umožní rychlé vyloučení rtuti močí. Jde např. o 2,3-disulfanylpropan-1-ol (dimerkaprol).
		\item Při nižší expozici se používá také dimethylcystein.
		\item Je také možné využít 2,3-disulfanyljantarovou kyselinu (DMSA).
	\end{itemize}
	\begin{figure}
		\adjincludegraphics[width=0.8\textwidth]{img/disulfanylpropanol.png}
	\end{figure}
	\vfill
}

\frame{
	\frametitle{}
	\begin{itemize}
		\item U rtuti byla poprvé pozorována supravodivost. Heike Kamerlingh-Onnes prováděl v roce 1911 měření odporu rtuti za nízkých teplot.\footnote[frame]{\href{https://home.cern/science/engineering/superconductivity}{Superconductivity}}
		\item Při teplotě 4,2 K pozoroval vymizení elektrického odporu.
		\item K chlazení rtuti využíval kapalné helium.
	\end{itemize}
	\begin{columns}
		\begin{column}{.6\textwidth}
		\begin{figure}
			\adjincludegraphics[width=.75\textwidth]{img/mercury-temp-res-1.png}
			\caption*{Závislost elektrického odporu rtuti na teplotě.}
		\end{figure}
		\end{column}
		\begin{column}{.4\textwidth}
		\begin{figure}
			\adjincludegraphics[width=.65\textwidth]{img/Kamerlingh_portret.jpg}
			\caption*{Heike Kammerling Onnes.\footnote[frame]{Zdroj: \href{https://commons.wikimedia.org/wiki/File:Kamerlingh_portret.jpg}{Commons}}}
		\end{figure}
	\end{column}
	\end{columns}
}

\section{Výskyt a získávání prvků}
\subsection{Zinek}
\frame{
	\frametitle{}
	\vfill
	\begin{itemize}
		\item Koncentrace zinku v zemské kůře je 75 ppm, mořská voda obsahuje přibližně 30~ppb zinku.
		\item Zinek patří mezi \textit{chalkofilní prvky},\footnote[frame]{Další chalkofilní prvky jsou Ag, As, Bi, Cd, Cu, Ga, Ge, Hg, In, Pb, S, Sb, Se, Sn, Te, Tl a Zn} tzn. že se ochotněji slučuje se sírou a těžšími chalkogeny než s kyslíkem.
		\item Hlavními minerály zinku jsou:
		\begin{itemize}
			\item Sfalerit, \ce{(Zn,Fe)S}
			\item Smithsonit, \ce{ZnCO3}
			\item Hemimorfit, \ce{Zn4(Si2O7)(OH)2.H2O}
			\item Wurtzit, \ce{(Zn,Fe)S}
		\end{itemize}
	\item Část zinku se získává i recyklací.
	\end{itemize}
	\vfill
}

\frame{
	\frametitle{}
	\vfill
	\textbf{Sfalerit}
	\begin{itemize}
		\item Kubický minerál, \ce{(Zn,Fe)S}, žluté až světle-hnědé barvy.\footnote[frame]{\href{https://mineraly.sci.muni.cz/sulfidy/sfalerit.html}{Sfalerit}}
		\item Důležitá ruda zinku.\footnote[frame]{\href{https://www.mindat.org/min-3727.html}{Sphalerite}}
		\item Jeden z nejběžnějších sulfidických minerálů.
		\item Patří mezi základní strukturní typy, každý atom v mřížce má tetraedrickou konfiguraci.
	\end{itemize}

	\begin{columns}
		\begin{column}{.4\textwidth}
			\begin{figure}
				\adjincludegraphics[height=.32\textheight]{img/Sphalerite.jpg}
				\caption*{Sfalerit, USA.\footnote[frame]{Zdroj: \href{https://commons.wikimedia.org/wiki/File:Sphalerite_-_Iron_Cap_mine,_Graham,_Arizona,_USA.jpg}{Ivar Leidus/Commons}}}
			\end{figure}
		\end{column}
		\begin{column}{.6\textwidth}
			\begin{figure}
				\adjincludegraphics[height=.32\textheight]{img/Sphalerite_polyhedra.png}
				\caption*{Krystalová struktura sfaleritu.\footnote[frame]{Zdroj: \href{https://commons.wikimedia.org/wiki/File:Sphalerite_polyhedra.png}{Solid State/Commons}}}
			\end{figure}
		\end{column}
	\end{columns}
	\vfill
}

\frame{
	\frametitle{}
	\vfill
	\textbf{Wurtzit}
	\begin{itemize}
		\item Hexagonální minerál, \ce{(Zn,Fe)S}, tmavě hnědé, oranžové nebo zelené barvy.\footnote[frame]{\href{https://mineraly.sci.muni.cz/sulfidy/wurtzit.html}{Wurtzit}}
		\item Polymorf sfaleritu.\footnote[frame]{\href{https://www.mindat.org/min-4318.html}{Wurtzite}}
		\item Patří mezi základní strukturní typy, každý atom v mřížce má tetraedrickou konfiguraci a jsou uspořádány ve sledu ABABABABAB.
	\end{itemize}

	\begin{columns}
		\begin{column}{.4\textwidth}
			\begin{figure}
				\adjincludegraphics[height=.32\textheight]{img/Wurtzite-245570.jpg}
				\caption*{Wurtzitu.\footnote[frame]{Zdroj: \href{https://commons.wikimedia.org/wiki/File:Wurtzite-245570.jpg}{Robert M. Lavinsky/Commons}}}
			\end{figure}
		\end{column}
		\begin{column}{.6\textwidth}
			\begin{figure}
				\adjincludegraphics[height=.32\textheight]{img/Wurtzite_polyhedra.png}
				\caption*{Krystalová struktura wurtzitu.\footnote[frame]{Zdroj: \href{https://commons.wikimedia.org/wiki/File:Wurtzite_polyhedra.png}{Solid State/Commons}}}
			\end{figure}
		\end{column}
	\end{columns}
	\vfill
}

\frame{
	\frametitle{}
	\vfill
	\textbf{Smithsonit}
	\begin{itemize}
		\item Trigonální minerál, \ce{ZnCO3}, bílé, nažloutlé až zelené barvy.\footnote[frame]{\href{https://mineraly.sci.muni.cz/karbonaty/smitsonit.html}{Smitsonit}}
		\item Pojmenován byl po anglickém chemikovi a mineralogovi Jamesi Smithsonovi.\footnote[frame]{\href{https://www.mindat.org/min-3688.html}{Smithsonite}}
	\end{itemize}

	\begin{columns}
		\begin{column}{.6\textwidth}
			\begin{figure}
				\adjincludegraphics[height=.3\textheight]{img/Smithsonite.jpg}
				\caption*{Smithsonit, Nové Mexiko.\footnote[frame]{Zdroj: \href{https://commons.wikimedia.org/wiki/File:Smithsonite_-_USGS_Mineral_Specimens_016.jpg}{Bureau of Mines/Commons}}}
			\end{figure}
		\end{column}
		\begin{column}{.4\textwidth}
			\begin{figure}
				\adjincludegraphics[height=.3\textheight]{img/Smithsonite-191703.jpg}
				\caption*{Smithsonit, Nové Mexiko.\footnote[frame]{Zdroj: \href{https://commons.wikimedia.org/wiki/File:Smithsonite-191703.jpg}{Robert M. Lavinsky/Commons}}}
			\end{figure}
		\end{column}
	\end{columns}
	\vfill
}

\frame{
	\frametitle{}
	\vfill
	\textbf{Hemimorfit}
	\begin{itemize}
		\item Orthorombický minerál, \ce{Zn4Si2O7(OH)2.H2O}, bezbarvé, zelené až modré barvy.\footnote[frame]{\href{https://mineraly.sci.muni.cz/sorosilikaty/hemimorfit.html}{Hemimorfit}}
		\item Často bývá asociován se smithsonitem.\footnote[frame]{\href{https://www.mindat.org/min-1860.html}{Hemimorphite}}
	\end{itemize}

	\begin{columns}
		\begin{column}{.5\textwidth}
			\begin{figure}
				\adjincludegraphics[height=.4\textheight]{img/Hemimorphit.jpg}
				\caption*{Hemimorfit, Čína.\footnote[frame]{Zdroj: \href{https://commons.wikimedia.org/wiki/File:Hemimorphit._China....2H1A7095WI.jpg}{Kora27/Commons}}}
			\end{figure}
		\end{column}
		\begin{column}{.5\textwidth}
			\begin{figure}
				\adjincludegraphics[height=.4\textheight]{img/Hemimorfit_Chiny.jpg}
				\caption*{Hemimorfit, Čína.\footnote[frame]{Zdroj: \href{https://commons.wikimedia.org/wiki/File:Hemimorfit,_Chiny.jpg}{Kluka/Commons}}}
			\end{figure}
		\end{column}
	\end{columns}
	\vfill
}

\frame{
	\frametitle{}
	\vfill
	\begin{itemize}
		\item Většina zinku se (mimo recyklace) získává ze sulfidických rud, které zpravidla obsahují příměsi mědi, olova a železa.\footnote[frame]{\href{https://youtu.be/LcU5miQbtNE}{Zinc Process Animation Video}}
		\item Ruda je rozdrcena a separována flotací, čímž se získá koncentrát sulfidu zinečnatého. Ten je následně pražen na vzduchu:
		\item \ce{2 ZnS + 3 O2 -> 2 ZnO + 2 SO2}
		\item Oxid siřičitý je dále zpracován na kyselinu sírovou, která je důležitým vedlejším produktem výroby zinku.
		\item Kovový zinek se vyrábí buď pyrometallurgicky nebo elektrolyticky.
		\item Pyrometallurgická metoda je založena na redukci uhlíkem nebo oxidem uhelnatým:
		\item \ce{2 ZnO + C -> 2 Zn + CO2}
		\item \ce{ZnO + CO -> Zn + CO2}
	\end{itemize}
	\vfill
}

\frame{
	\frametitle{}
	\vfill
	\begin{columns}
		\begin{column}{.65\textwidth}
			\begin{itemize}
				\item Elektrolytická metoda je založena na rozpuštění oxidu v kyselině sírové:
				\item \ce{ZnO + H2SO4 -> ZnSO4 + H2O}
				\item Vzniklý roztok síranu je elektrolyzován:
				\item \ce{2 ZnSO4 + 2 H2O -> 2 Zn + 2 H2SO4 + O2}
				\item Kyselina sírová se tímto krokem regeneruje a vrací zpět na začátek procesu.
				\item Zinek se vylučuje na hliníkové fólii, z které je odloupnut a roztaven v indukční peci.
				\item Z taveniny se odlévají ingoty, příp. se upravují vlastnosti kovu sléváním s jinými prvky.
			\end{itemize}
		\end{column}
		\begin{column}{.4\textwidth}
			\begin{figure}
				\adjincludegraphics[height=.4\textheight]{img/Zinc_world_production.png}
				\caption*{Celosvětový objem výroby zinku.\footnote[frame]{Zdroj: \href{https://commons.wikimedia.org/wiki/File:Zinc_world_production.svg}{Con-struct/Commons}}}
			\end{figure}
		\end{column}
	\end{columns}
	\vfill
}

\frame{
	\frametitle{}
	\vfill
	\begin{figure}
		\adjincludegraphics[width=\textwidth]{img/Primary_zinc_smelting_flowchart.png}
		\caption*{Schéma výroby zinku. Horní část popisuje pyrometallurgický proces a spodní elektrolytický.\footnote[frame]{Zdroj: \href{https://commons.wikimedia.org/wiki/File:Primary_zinc_smelting_flowchart_-vector.svg}{US Environmental Protection Agency/Commons}}}
	\end{figure}
	\begin{itemize}
		\item Těžba a výroba zinku je ekologicky velmi problematická, do ovzduší se dostává velké množství oxidu siřičitého a kadmia.\footnote[frame]{\href{https://doi.org/10.1007/BF00770598}{Primary minerals of Zn-Pb mining and metallurgical dumps and their environmental behavior at Plombières, Belgium}}
	\end{itemize}
	\vfill
}

\subsection{Kadmium}
\frame{
	\frametitle{}
	\vfill
	\begin{itemize}
		\item Kadmium je výrazně vzácnější než zinek, jeho koncentrace v zemské kůře je asi 0,1~ppm.
		\item Jediný významný minerál kadmia je greenockit, CdS.
		\item Kadmium se získává jako vedlejší produkt při výrobě zinku a také olova a mědi.
		\item Vakuově se oddestilovává ze zinku nebo se při elektrolytické výrobě sráží jako síran, \ce{CdSO4}.
		\item Celosvětový objem roční výroby kadmia se blíží k 20 000 tunám.
	\end{itemize}
	\begin{figure}
		\adjincludegraphics[height=0.3\textheight]{img/2005cadmium.png}
		\caption*{Světová produkce kadmia v roce 2005.\footnote[frame]{Zdroj: \href{https://commons.wikimedia.org/wiki/File:2005cadmium.PNG}{Anwar saadat/Commons}}}
	\end{figure}
	\vfill
}

\frame{
	\frametitle{}
	\vfill
	\textbf{Greenockit}
	\begin{itemize}
		\item Hexagonální minerál, \ce{CdS}, žlutá, hnědá nebo načervenalá barva.\footnote[frame]{\href{https://mineraly.sci.muni.cz/sulfidy/greenockit.html}{Greenockit}}
		\item Za nižší teplot je izostrukturní se sfaleritem, za vyšší s wurtzitem.\footnote[frame]{\href{https://www.mindat.org/min-1746.html}{Greenockite}}
		\item Dříve se využíval jako žlutý pigment, kadmiová žluť.\footnote[frame]{\href{https://www.winsornewton.com/row/articles/colours/spotlight-on-cadmium-yellow/}{Colour story: cadmium yellow}}
	\end{itemize}

	\begin{columns}
		\begin{column}{.5\textwidth}
			\begin{figure}
				\adjincludegraphics[height=.35\textheight]{img/Greenockite-259580.jpg}
				\caption*{Greenockit, Namíbie.\footnote[frame]{Zdroj: \href{https://commons.wikimedia.org/wiki/File:Greenockite-259580.jpg}{Christian Rewitzer/Commons}}}
			\end{figure}
		\end{column}
		\begin{column}{.5\textwidth}
			\begin{figure}
				\adjincludegraphics[height=.35\textheight]{img/Greenockite-Hemimorphite.jpg}
				\caption*{Greenockit, USA.\footnote[frame]{Zdroj: \href{https://commons.wikimedia.org/wiki/File:Greenockite-Hemimorphite-Sphalerite-285103.jpg}{Robert M. Lavinsky/Commons}}}
			\end{figure}
		\end{column}
	\end{columns}
	\vfill
}

\subsection{Rtuť}
\frame{
	\frametitle{}
	\vfill
	\begin{columns}
		\begin{column}{.65\textwidth}
			\begin{itemize}
				\item Rtuť je poměrně vzácná, její koncentrace v zemské kůře je jen 0,08~ppm.
				\item Známe 88 minerálů obsahujících rtuť.\footnote[frame]{\href{https://www.mindat.org/element/Mercury}{The mineralogy of Mercury}}
				\item Velmi vzácně se nachází v ryzím stavu, nejběžnější minerál je cinabarit, \ce{HgS}.
				\item Celosvětová produkce rtuti v roce 2019 byla necelých 4 000 tun.\footnote[frame]{\href{https://www.usgs.gov/centers/nmic/mercury-statistics-and-information}{Mercury Statistics and Information}}
				\item Hlavními producenty jsou Čína a Mexiko.
			\end{itemize}
		\end{column}
		\begin{column}{.4\textwidth}
			\begin{figure}
				\adjincludegraphics[width=\textwidth]{img/HgProductionPrice.png}
				\caption*{Vývoj ceny a objemu výroby rtuti.\footnote[frame]{Zdroj: \href{https://commons.wikimedia.org/wiki/File:HgProductionPrice.png}{USGS/Commons}}}
			\end{figure}
		\end{column}
	\end{columns}
	\vfill
}

\frame{
	\frametitle{}
	\vfill
	\textbf{Cinabarit}
	\begin{itemize}
		\item Trigonální minerál, \ce{HgS}, červené barvy.\footnote[frame]{\href{https://mineraly.sci.muni.cz/sulfidy/cinabarit.html}{Cinabarit}}
		\item Zdroj pro průmyslovou výrobu rtuti.\footnote[frame]{\href{https://www.mindat.org/min-1052.html}{Cinnabar}}
		\item Ve středověku se využíval jako červený pigment.
	\end{itemize}

	\begin{columns}
		\begin{column}{.5\textwidth}
			\begin{figure}
				\adjincludegraphics[height=.35\textheight]{img/Cinnabar.jpg}
				\caption*{Cinabarit, USA.\footnote[frame]{Zdroj: \href{https://commons.wikimedia.org/wiki/File:Cinnabar_(New_Almaden,_California,_USA)_(18824306236).jpg}{James St. John/Commons}}}
			\end{figure}
		\end{column}
		\begin{column}{.5\textwidth}
			\begin{figure}
				\adjincludegraphics[height=.35\textheight]{img/Cinnabar-3d6c.jpg}
				\caption*{Cinabarit, Čína.\footnote[frame]{Zdroj: \href{https://commons.wikimedia.org/wiki/File:Cinnabar-3d6c.jpg}{Robert M. Lavinsky/Commons}}}
			\end{figure}
		\end{column}
	\end{columns}
	\vfill
}

\frame{
	\frametitle{}
	\vfill
	\begin{figure}
		\adjincludegraphics[height=.7\textheight]{img/Chinese_carved_cinnabar_lacquerware.jpg}
		\caption*{Čínské vázy barvené cinabaritem.\footnote[frame]{Zdroj: \href{https://en.wikipedia.org/wiki/File:Chinese_carved_cinnabar_lacquerware.jpg}{Danieliness}}}
	\end{figure}
	\vfill
}

\frame{
	\frametitle{}
	\vfill
	\textbf{Kalomel}
	\begin{itemize}
		\item Tetragonální minerál, \ce{Hg2Cl2}, červené barvy.\footnote[frame]{\href{https://www.mindat.org/min-869.html}{Calomel}}
		\item Využívá se pro konstrukci kalomelových referenčních elektrod.
		\item Název kalomel je odvozen z řeckých slov \textit{kalós} (krásný) a \textit{mélas} (černý), pravděpodobně z důvodu černé barvy vznikající reakcí amoniaku s kalomelem.
		\item Čistý \ce{Hg2Cl2} je bílý.
	\end{itemize}

	\begin{columns}
		\begin{column}{.5\textwidth}
			\begin{figure}
				\adjincludegraphics[height=.27\textheight]{img/Calomel-154907.jpg}
				\caption*{Kalomel, Texas.\footnote[frame]{Zdroj: \href{https://commons.wikimedia.org/wiki/File:Calomel-154907.jpg}{Robert M. Lavinsky/Commons}}}
			\end{figure}
		\end{column}
		\begin{column}{.5\textwidth}
			\begin{figure}
				\adjincludegraphics[height=.27\textheight]{img/Calomel-222734.jpg}
				\caption*{Kalomel a terlinguait (\ce{Hg4Cl2O2}), Texas.\footnote[frame]{Zdroj: \href{https://commons.wikimedia.org/wiki/File:Calomel,_Terlinguaite-222734.jpg}{Kelly Nash/Commons}}}
			\end{figure}
		\end{column}
	\end{columns}
	\vfill
}

\frame{
	\frametitle{}
	\vfill
	\begin{columns}
		\begin{column}{.65\textwidth}
			\begin{itemize}
				\item Dříve se rtuť získávala zahříváním cinabaritu na vzduchu a kondenzací par.
				\item \ce{HgS + O2 ->[600 $^\circ$C] Hg + SO2}
				\item Výhodnější je provést redukci železem nebo páleným vápnem:
				\item \ce{HgS + Fe -> Hg + FeS}
				\item \ce{4 HgS + 4 CaO -> 4 Hg + 3 CaS + CaSO4}
				\item Zbytky kovů lze odstranit oxidací vzduchem, kdy dochází ke vzniku oxidů, které přecházejí do strusky.
				\item Rtuť lze dále přečistit destilací.
			\end{itemize}
		\end{column}
		\begin{column}{.5\textwidth}
			\begin{figure}
				\adjincludegraphics[height=.35\textheight]{img/Mercury-distillation.png}
				\caption*{Destilace rtuti.\footnote[frame]{Zdroj: \href{https://commons.wikimedia.org/wiki/File:\%D0\%9E\%D1\%87\%D0\%B8\%D1\%81\%D1\%82\%D0\%BA\%D0\%B0_\%D1\%80\%D1\%82\%D1\%83\%D1\%82\%D0\%B8_\%D0\%BF\%D0\%B5\%D1\%80\%D0\%B5\%D0\%B3\%D0\%BE\%D0\%BD\%D0\%BA\%D0\%BE\%D0\%B9_\%D0\%B2_\%D1\%82\%D0\%BE\%D0\%BA\%D0\%B5_\%D0\%B3\%D0\%B0\%D0\%B7\%D0\%B0.png}{Commons}}}
			\end{figure}
		\end{column}
	\end{columns}
	\vfill
}

\section{Využití prvků}
\subsection{Zinek}
\frame{
	\frametitle{}
	\vfill
	\begin{itemize}
		\item Velká část zinku se využívá jako antikorozní ochrana, protože je jeho reaktivita vyšší než reaktivita běžných ocelí.
		\item Jeho korozí vzniká povrchová vrstva o složení \ce{Zn5(OH)6(CO3)2}.
		\item Zinek se na povrch kovů nanáší elektrochemicky nebo z taveniny.
		\item Díky zápornému elektrodovému potenciálu ($-$0,76~V) se využívá jako anoda v bateriích.
	\end{itemize}
	\begin{figure}
		\adjincludegraphics[height=0.4\textheight]{img/Zincbattery.png}
		\caption*{Zinková baterie.\footnote[frame]{Zdroj: \href{https://commons.wikimedia.org/wiki/File:Zincbattery.png}{Jacek FH/Commons}}}
	\end{figure}
	\vfill
}

\frame{
	\frametitle{}
	\vfill
	\begin{itemize}
		\item \textit{Mosaz} je slitina zinku a mědi.
		\item Používá se už od starověku.
		\item Vyrábí se sléváním mědi a zinku, tady je ale komplikací těkavost zinku.
		\item Mosazi se také legují, zpravidla Fe, Al, Mn, Ni a Sn.
	\end{itemize}
	\begin{figure}
		\adjincludegraphics[height=0.4\textheight]{img/Brass_leopard.jpg}
		\caption*{Bronzový leopard.\footnote[frame]{Zdroj: \href{https://commons.wikimedia.org/wiki/File:Brass_leopard.jpg}{ZSM/Commons}}}
	\end{figure}
	\vfill
}

\frame{
	\frametitle{}
	\vfill
	\begin{columns}
		\begin{column}{.5\textwidth}
			\begin{figure}
				\adjincludegraphics[height=0.6\textheight]{img/Torun_kosciol_garn_dach.jpg}
				\caption*{Střecha z pozinkovaného plechu.\footnote[frame]{Zdroj: \href{https://commons.wikimedia.org/wiki/File:Torun_kosciol_garn_dach.jpg}{Pko/Commons}}}
			\end{figure}
		\end{column}
		\begin{column}{.5\textwidth}
			\begin{figure}
				\adjincludegraphics[height=0.6\textheight]{img/Brassdoor.jpg}
				\caption*{Mosazné dveře.\footnote[frame]{Zdroj: \href{https://commons.wikimedia.org/wiki/File:A_brassdoor_in_the_City_Palace_complex,_Jaipur.jpg}{Fernando Nunes/Commons}}}
			\end{figure}
		\end{column}
	\end{columns}
	\vfill
}

\frame{
	\frametitle{}
	\vfill
	\begin{itemize}
		\item Organozinečnaté sloučeniny se využívají v organické katalýze.
	\end{itemize}
	\begin{figure}
		\adjincludegraphics[height=0.6\textheight]{img/DiphenylzincCarbonylAddition.png}
		\caption*{Adice difenylzinku na karbonyl.\footnote[frame]{Zdroj: \href{https://commons.wikimedia.org/wiki/File:DiphenylzincCarbonylAddition.png}{V8rik/Commons}}}
	\end{figure}
	\vfill
}

\subsection{Kadmium}
\frame{
	\frametitle{}
	\vfill
	\begin{itemize}
		\item Kadmium je součástí Ni--Cd akumulátorů.
		\item První baterie tohoto typu byla vyrobena už roku 1899.
		\item Kladná elektroda je tvořena NiO(OH) a záporná kovovým kadmiem.
		\item Vybíjení akumulátoru:
		\item \ce{Cd + 2 NiO(OH) + 2 H2O -> Cd(OH)2 + 2 Ni(OH)2}
	\end{itemize}
	\begin{figure}
		\adjincludegraphics[height=0.4\textheight]{img/Varta_batteries.jpg}
		\caption*{Ni--Cd akumulátory.\footnote[frame]{Zdroj: \href{https://commons.wikimedia.org/wiki/File:Varta_batteries.JPG}{Tukka/Commons}}}
	\end{figure}
	\vfill
}

\frame{
	\frametitle{}
	\vfill
	\begin{itemize}
		\item Významnou sloučeninou je tellurid kademnatý (\ce{CdTe}), má strukturu sfaleritu.
		\item Využívá se při konstrukci solárních článků, výhodou jsou malé náklady. Tenkovrstvé CdTe články patří mezi levnější a jsou velmi rozšířené.
	\end{itemize}
	\begin{columns}
		\begin{column}{.5\textwidth}
			\begin{figure}
				\adjincludegraphics[height=0.4\textheight]{img/Photodetector.jpg}
				\caption*{CdTe fotodetektor z CD--ROM.\footnote[frame]{Zdroj: \href{https://commons.wikimedia.org/wiki/File:CD-ROM_Photodetector.jpg}{H0dges/Commons}}}
			\end{figure}
		\end{column}
		\begin{column}{.5\textwidth}
			\begin{figure}
				\adjincludegraphics[height=0.4\textheight]{img/NREL_Array.jpg}
				\caption*{Tenkovrstvé CdTe solární články.\footnote[frame]{Zdroj: \href{https://commons.wikimedia.org/wiki/File:NREL_Array.jpg}{NREL/Commons}}}
			\end{figure}
		\end{column}
	\end{columns}
	\vfill
}

\frame{
	\frametitle{}
	\vfill
	\begin{itemize}
			\item Slitina s rtutí se využívá pro konstrukci MCT detektorů pro infračervenou spektroskopii.\footnote[frame]{\href{http://irassociates.com/index.php?page=ln2-cooled}{LN2 Cooled HgCdTe Detectors}}
		\item MCT -- Mercury-Cadmium-Telluride
	\end{itemize}
	\begin{columns}
		\begin{column}{.5\textwidth}
			\adjincludegraphics[width=\textwidth]{img/TGIR2.jpg}
		\end{column}
		\begin{column}{.5\textwidth}
			\adjincludegraphics[width=\textwidth]{img/TGIR3.jpg}
		\end{column}
	\end{columns}
	\vfill
}

\subsection{Rtuť}
\frame{
	\frametitle{}
	\vfill
	\begin{itemize}
		\item Rtuť se využívá v medicíně, ale kvůli její toxicitě se poptávka po ní snižuje.
		\item V zubařství se využívají amalgámové plomby (Hg 50~\%, Ag 22--32~\%, Sn 14~\% a Zn 8~\%).\footnote[frame]{\href{https://www.wikiskripta.eu/w/Amalgam}{Amalgám}}
		\item Amalgámy se připravují smísením rtuti s práškovými kovy.
		\item Vysoké teplotní roztažnosti se využívá v lékařských teploměrech.
	\end{itemize}
	\begin{columns}
		\begin{column}{.5\textwidth}
			\begin{figure}
				\adjincludegraphics[height=0.32\textheight]{img/Maximum_thermometer.jpg}
			\caption*{Rtuťový teploměr.\footnote[frame]{Zdroj: \href{https://commons.wikimedia.org/wiki/File:Maximum_thermometer_close_up_2.JPG}{CambridgeBayWeather/Commons}}}
			\end{figure}
		\end{column}
		\begin{column}{.5\textwidth}
			\begin{figure}
				\adjincludegraphics[height=0.32\textheight]{img/Amalgam.jpg}
				\caption*{Amalgámová plomba.\footnote[frame]{Zdroj: \href{https://commons.wikimedia.org/wiki/File:Amalgam.jpg}{Ulrich Birkhoff/Commons}}}
			\end{figure}
		\end{column}
	\end{columns}
	\vfill
	\vfill
}

\frame{
	\frametitle{}
	\vfill
	\begin{itemize}
		\item Elektrické vodivosti a tekutosti rtuti se využívá v různých modifikacích elektronických spínačů.\footnote[frame]{\href{http://danyk.cz/hg_sp.html}{Rtuťové spínače}}
		\item Mohou sloužit i jako snímače polohy.
	\end{itemize}
	\begin{figure}
		\adjincludegraphics[height=0.5\textheight]{img/Mercury_Switch.jpg}
		\caption*{Rtuťový spínač.\footnote[frame]{Zdroj: \href{https://commons.wikimedia.org/wiki/File:Mercury_Switch_without_housing.jpg}{Medvedev/Commons}}}
	\end{figure}
	\vfill
}

\frame{
	\frametitle{}
	\vfill
	\begin{itemize}
		\item Polarografie je elektrochemická metoda, kterou objevil Jaroslav Heyrovský. Za tento objev dostal v roce 1959 Nobelovu cenu za chemii.\footnote[frame]{\href{https://www.nobelprize.org/prizes/chemistry/1959/summary/}{The Nobel Prize in Chemistry 1959}}
		\item Jde o kvantitativní i kvalitativní metodu.
		\item Kvalitu určuje poloha vlny a kvantitu její výška.
		\item Analýza probíhá na rtuťové kapce.
		\item V roztoku nesmí být přítomen kyslík.
	\end{itemize}

	\begin{columns}
		\begin{column}{.4\textwidth}
			\begin{figure}
				\adjincludegraphics[height=.3\textheight]{img/Heyrovsky_Jaroslav.jpg}
				\caption*{Jaroslav Heyrovský.\footnote[frame]{Zdroj: \href{https://commons.wikimedia.org/wiki/File:Heyrovsky_Jaroslav.jpg}{archiv ÚFCH J.Heyrovského AV ČR, v.v.i./Commons}}}
			\end{figure}
		\end{column}
		\begin{column}{.6\textwidth}
			\begin{figure}
				\adjincludegraphics[height=.3\textheight]{img/Polarogramm.jpg}
				\caption*{Polarogram.\footnote[frame]{Zdroj: \href{https://commons.wikimedia.org/wiki/File:Beispiel-Polarogramm.JPG}{Bashir001/Commons}}}
			\end{figure}
		\end{column}
	\end{columns}
	\vfill
}

\frame{
	\frametitle{}
	\vfill
	\begin{figure}
		\adjincludegraphics[width=.9\textwidth]{img/Polarografie.jpg}
		\caption*{Heyerovského polarograf.\footnote[frame]{Zdroj: \href{https://commons.wikimedia.org/wiki/File:Polarografie_-_P\%C5\%99\%C3\%ADb\%C4\%9Bh_kapky_foto_Pavl\%C3\%ADna_J\%C3\%A1chimov\%C3\%A1\%2C_AV_\%C4\%8CR_\%281\%29.jpg}{Akademie věd České republiky/Commons}}}
	\end{figure}
	\vfill
}

\frame{
	\frametitle{}
	\vfill
	\begin{columns}
		\begin{column}{.7\textwidth}
			\begin{itemize}
				\item \textbf{Kalomelová elektroda} je srovnávací (referentní) elektrodou, je konstrukčně jednodušší než vodíková elektroda.
				\item Je tvořená rtutí pokrytou vrstvou kalomelu v roztoku KCl.
				\item Hg, \ce{Hg2Cl2} \ce{|} KCl
				\item Na fázovém rozhraní se ustavuje rovnováha:
				\item \ce{Hg2Cl2 + 2 e- <=> 2 Hg + 2 Cl-}
				\item Potenciál je dán koncentrací chloridových aniontů:
				\item $E = E^0(Hg_2Cl_2) - 0,059 \log a(Cl^-)$
				\item $E = 0,268 - 0,059 \log a(Cl^-)$
			\end{itemize}
		\end{column}
		\begin{column}{.3\textwidth}
			\begin{figure}
				\adjincludegraphics[width=1.2\textwidth]{img/Calomol_electrode_image.jpg}
				\caption*{Kalomelová elektroda.\footnote[frame]{Zdroj: \href{https://commons.wikimedia.org/wiki/File:Calomol_electrode_image.jpg}{Janakiraman janaki/Commons}}}
			\end{figure}
		\end{column}
	\end{columns}
	\vfill
}

\frame{
	\frametitle{}
	\vfill
	\begin{itemize}
		\item Merkurimetrie je metoda odměrné analýzy, konkrétně komplexometrické odměrné analýzy.
		\item Jako odměrný roztok se využívá vodný roztok \ce{Hg(NO3)2}.
		\item Využívá se ke stanovení koncentrace halogenidů a kyanidů.
		\item Jde o levnější alternativu argentometrie.
		\item Při titraci dochází ke vzniku koordinačních sloučenin:
		\item \ce{Hg(NO3)2 + 4 NaCl -> Na2[HgCl4] + 2 NaNO3}
		\item Následně dochází k rozpadu komplexů:
		\item \ce{[HgCl4]^{2-} -> [HgCl3]^{-} -> HgCl2 -> HgCl^+}
		\item Jako indikátor se využívá nitroprussid sodný (\ce{Na2[Fe(CN)5NO]}) nebo difenylkarbazon.
	\end{itemize}
	\vfill
}

\frame{
	\frametitle{}
	\vfill
	\begin{columns}
		\begin{column}{.5\textwidth}
		\begin{itemize}
		\item Rtuť se využívá při elektrolýze NaCl, kdy se vznikající sodík váže do amalgámu a následně reaguje s vodou za vzniku NaOH, druhým produktem je chlor.
		\item Rtuť se také používá v rtuťových manometrech.
		\item Kapalná zrcadla do teleskopů.\footnote[frame]{\href{https://physicsfootnotes.com/footnotes/liquid-mirror-telescopes/}{Liquid Mirror Telescopes}}
		\item Stopy rtuti se nacházejí v zářivkách.\footnote[frame]{\href{https://www.youtube.com/watch?v=NY3Q_Lt7XYw}{Video - Je rtuť opravdu tolik nebezpečná?}}
	\end{itemize}
	\end{column}
	\begin{column}{.5\textwidth}
		\begin{figure}
			\adjincludegraphics[width=.75\textwidth]{img/Barometer_mercury.jpg}
			\caption*{Rtuťový manometr.\footnote[frame]{Zdroj: \href{https://commons.wikimedia.org/wiki/File:Barometer_mercury_column_hg.jpg}{Hannes Grobe/Commons}}}
		\end{figure}
	\end{column}
	\end{columns}
	\vfill
}

\section{Sloučeniny}
\subsection{Halogenidy}
\frame{
	\frametitle{}
	\vfill
	Ve třetím řádku jsou T$_t$ a T$_v$ [$^\circ$C]
	\begin{tabular}{|c|c|c|c|}
		\hline
		\textbf{Fluoridy} & \textbf{Chloridy} & \textbf{Bromidy} & \textbf{Jodidy} \\\hline
		\ce{ZnF2} & \ce{ZnCl2} & \ce{ZnBr2} & \ce{ZnI2} \\
		bílý & bílý & bílý & bílý \\
		872, 1500 & 275, 756 & 394, 702 & 446, rozkl. $>$700 \\\hline
		\ce{CdF2} & \ce{CdCl2} & \ce{CdBr2} & \ce{CdI2} \\
		bílý & bílý & světle žlutý & bílý \\
		1049, 1748 & 568, 980 & 566, 863 & 388, 787 \\\hline
		\ce{HgF2} & \ce{HgCl2} & \ce{HgBr2} & \ce{HgI2} \\
		bílý & bílý & bílý & červený, žlutý \\
		rozkl. $>$645 & 280, 303 & 238, 318 & 257, 351 \\\hline
		\ce{Hg2F2} & \ce{Hg2Cl2} & \ce{Hg2Br2} & \ce{Hg2I2} \\
		žlutý & bílý & bílý & žlutý \\
		rozkl. $>$570 & subl. 383 & subl. 345 & subl. 140 \\\hline
	\end{tabular}
	\vfill
}

\frame{
	\frametitle{}
	\vfill
	\begin{itemize}
		\item Bezvodé fluoridy lze připravit reakcí HF se Zn nebo fluoru s~Cd a Hg.
		\item Ostatní zinečnaté a kademnaté halogenidy jsou hygroskopické a~dobře rozpustné ve vodě.
		\begin{itemize}
			\item Halogenidy zinečnaté mají rozpustnost asi 400~g na 100~g vody.
			\item Halogenidy kademnaté mají rozpustnost asi 100~g na 100~g vody.
		\end{itemize}
		\item Vysoká rozpustnost je dána tvorbou komplexních iontů:
		\item \ce{ZnCl2 + 4 H2O -> [Zn(H2O)4]Cl2}
		\item U halogenidů rtuťnatých pozorujeme výraznější kovalentní charakter než u lehčích analog.
		\item Lze je snadno připravit z prvků.
		\item \ce{HgCl2} se skládá z lineárních molekul \ce{Cl-Hg-Cl}.
	\end{itemize}
	\vfill
}

\frame{
	\frametitle{}
	\vfill
	\begin{figure}
		\adjincludegraphics[width=\textwidth]{img/HgCl2-xtal.png}
		\caption*{Krystalová struktura \ce{HgCl2}\footnote[frame]{\href{https://commons.wikimedia.org/wiki/File:Mercury(II)-chloride-xtal-1980-3D-balls.png}{Zdroj: Ben Mills/Commons}}}
	\end{figure}
	\vfill
}

\frame{
	\frametitle{}
	\vfill
	\begin{itemize}
		\item \ce{HgI2} vykazuje termochromismus.\footnote[frame]{Změna barvy s teplotou} Přechod nastává při teplotě 126~°C.
		\item Červená $\alpha$-forma přechází na žlutou $\beta$-formu.
		\item Známe ještě třetí formu, ta je oranžová, vzniká rekrystalizací.
		\item Je také metastabilní a samovolně přechází na červenou formu.
		\item V přírodě se vyskytuje jako vzácný minerál coccinit.\footnote[frame]{\href{https://www.mindat.org/min-1100.html}{Coccinite}}
	\end{itemize}

	\begin{figure}
		\adjincludegraphics[height=.35\textheight]{img/Mercury_iodide.jpg}
		\caption*{Jodid rtuťnatý.\footnote[frame]{Zdroj: \href{https://commons.wikimedia.org/wiki/File:Mercury_iodide.jpg}{W. Oelen/Commons}}}
	\end{figure}
	\vfill
}

\frame{
	\frametitle{}
	\vfill
	\begin{itemize}
		\item Chlorid rtuťnatý je velmi málo rozpustný ve vodě (0,006~g/100~g), ale dobře se rozpouští v roztocích alkalických iodidů. Dochází totiž ke vzniku komplexních sloučenin:\footnote[frame]{\href{https://edu.ceskatelevize.cz/video/3404-vznik-nesslerova-cinidla}{Vznik Nesslerova činidla}}
		\item \ce{HgI2 + 2 KI -> K2[HgI4]}
		\item Podobně se chová i chlorid rtuťnatý:
		\item \ce{HgCl2 + 4 KI -> K2[HgI4] + 2 KCl}
		\item Vzniklý tetrajodidortuťnatan draselný se označuje jako \textit{Nesslerovo činidlo} a používá se k důkazu přítomnosti amoniaku v bazickém prostředí:
		\item \ce{NH4Cl + 2 K2[HgI4] + 4 KOH -> HgO.Hg(NH2)I v + 7 KI + KCl + 3 H2O}
		\item Vzniklý komplex se označuje jako \textit{Millonova báze} a má žlutou nebo hnědou barvu.
	\end{itemize}
	\vfill
}

\subsection{Zinek}
\frame{
	\frametitle{}
	\vfill
	\begin{columns}
	\begin{column}{.75\textwidth}
	\begin{itemize}
		\item Na vlhkém vzduchu rychle ztrácí lesk.
		\item Je reaktivnější než měď.
		\item Reaguje s kyslíkem, sírou a fosforem. Při zahřívání s halogeny.
		\item Z minerálních kyselin uvolňuje vodík.
		\item Má redukční vlastnosti.
		\item Preferuje oxidační číslo II, ale je známo i několik sloučenin v oxidačním čísle I.
		\item První připravenou sloučeninou \ce{Zn^I} byl dekamethyldizinkonocen. Byl získán reakcí zinkonocenu s diethylzinkem. Obsahuje vazbu \ce{Zn-Zn}.\footnote[frame]{\href{https://doi.org/10.1126/science.1101356}{Decamethyldizincocene, a Stable Compound of Zn(I) with a Zn-Zn Bond}}
		\item Později se povedlo tuto sloučeninu připravit i redukcí zinkonocenu pomocí KH.\footnote[frame]{\href{https://doi.org/10.1021/ja0668217}{Zinc-Zinc Bonded Zincocene Structures}}
	\end{itemize}
	\end{column}
	\begin{column}{.3\textwidth}
		\begin{figure}
			\adjincludegraphics[width=1\textwidth]{img/Decamethyldizincocene_xtal.png}
			\caption*{Dekamethyldizinkonocen.\footnote[frame]{Zdroj: \href{https://commons.wikimedia.org/wiki/File:Decamethyldizincocene_xtal.png}{Rifleman 82/Commons}}}
		\end{figure}
	\end{column}
	\end{columns}
	\vfill
}

\frame{
	\frametitle{}
	\vfill
	\begin{figure}
		\adjincludegraphics[width=1\textwidth]{img/ZnI-konocene.png}
		\caption*{Syntéza dekamethylzinconocenu.\footnote[frame]{\href{https://doi.org/10.1126/science.1101356}{Decamethyldizincocene, a Stable Compound of Zn(I) with a Zn-Zn Bond}}}
	\end{figure}
	\vfill
}

\frame{
	\frametitle{}
	\vfill
	\begin{itemize}
		\item Elektronová konfigurace iontu \ce{Zn+} je 5d$^{10}$ 4s$^1$.
		\item Sloučeniny \ce{Zn^I} disproporcionují za vzniku zinečnaté soli a kovového zinku.
		\item \ce{Zn$_2^{2+}$ -> Zn^{2+} + Zn}
		\item Zinečnaté soli jsou stabilní, v bazických roztocích se srážejí za vzniku amorfního \ce{Zn(OH)2}. V silně bazických roztocích se rozpouští za vzniku \ce{[Zn(OH)4]^{2-}}.
		\item \ce{Zn^{2+} + 2 OH- -> Zn(OH)2 v ->[OH-] [Zn(OH)4]^{2-}}
		\item Známe všechny halogenidy i chalkogenidy zinečnaté. Chalkogenidy se využívají v elektronice a optice.
		\item Zinečnatý ion má elektronovou konfiguraci d$^{10}$, proto jsou sloučeniny bezbarvé a diamagnetické.
	\end{itemize}
	\vfill
}

\frame{
	\frametitle{}
	\vfill
	\begin{columns}
		\begin{column}{.65\textwidth}
			\begin{itemize}
				\item \textit{Oxid zinečnatý}, ZnO, se označuje jako \textit{zinková běloba}. Využívá se jako bílý pigment.
				\item Je amfoterní, rozpouští se v kyselinách i hydroxidech.
				\item Lze jej připravit termickým rozkladem hydroxidu, uhličitanu nebo dusičnanu zinečnatého:
				\item \ce{2 Zn(NO3)2 -> 2 ZnO + 4 NO2 + O2}
				\item Průmyslově se vyrábí spalováním kovového zinku:
				\item \ce{2 Zn + O2 -> 2 ZnO}
			\end{itemize}
		\end{column}
		\begin{column}{.4\textwidth}
			\begin{figure}
				\adjincludegraphics[width=\textwidth]{img/Zinc_oxide.jpg}
				\caption*{Oxid zinečnatý.\footnote[frame]{Zdroj: \href{https://commons.wikimedia.org/wiki/File:Zinc_oxide.jpg}{Walkerma/Commons}}}
			\end{figure}
		\end{column}
	\end{columns}
	\vfill
}

\frame{
	\frametitle{}
	\vfill
	\begin{itemize}
		\item Uhličitan zinečnatý, \ce{ZnCO3}, je nerozpustná bílá látka. V přírodě se vyskytuje jako minerál \textit{smithsonit}.
		\item Zahříváním se mění na bazický uhličitan zinečnatý, \ce{Zn5(CO3)2(OH)6}.
		\item Krystalová struktura je podobná uhličitanu vápenatému, zinečnaté ionty mají oktaedrickou konfiguraci a uhličitany jsou vázány k šesti zinkům, atomy kyslíku mají koordinační číslo tři.
		\item Využívá se v dermatologii jako součást kalamínových mastí.\footnote[frame]{\href{https://www.sciencedirect.com/topics/pharmacology-toxicology-and-pharmaceutical-science/zinc-carbonate}{Zinc Carbonate}}
	\end{itemize}
	\begin{figure}
		\adjincludegraphics[width=.7\textwidth]{img/ZnCO3.png}
		\caption*{Uhličitan zinečnatý.\footnote[frame]{Zdroj: \href{https://commons.wikimedia.org/wiki/File:Uhli\%C4\%8Ditan_zine\%C4\%8Dnat\%C3\%BD.PNG}{Ondřej Mangl/Commons}}}
	\end{figure}
	\vfill
}

\subsection{Kadmium}
\frame{
	\frametitle{}
	\vfill
	\begin{itemize}
		\item Chemicky je kadmium podobné zinku, vytváří sloučeniny v oxidačním čísle II, velmi výjimečně i I.
		\item Na vzduchu hoří za vzniku amorfního \ce{CdO}.
		\item V minerálních kyselinách se rozpouští za vzniku kademnatých solí.
		\item Redukcí chloridu kademnatého v tavenině kadmia vzniká kation \ce{Cd$_2^{2+}$} s vazbou \ce{Cd-Cd}:
		\item \ce{CdCl2 + Cd -> Cd2Cl2}
		\item Následnou reakcí s \ce{AlCl3} vzniká \ce{Cd2(AlCl4)2}:\footnote[frame]{\href{https://doi.org/10.1021/ja01462a016}{Stabilization of the Cadmium(I) Oxidation State.}}
		\item \ce{Cd2Cl2 + 2 AlCl3 -> Cd2(AlCl4)2}
		\item Molekula obsahuje jednotky \ce{Cd2Cl6} a tetraedry \ce{AlCl4} spojené vrcholy.
	\end{itemize}
	\vfill
}

\frame{
	\frametitle{}
	\vfill
	\begin{figure}
		\adjincludegraphics[height=.65\textheight]{img/Cd-AlCl4.png}
		\caption*{Krystalová struktura \ce{Cd2(AlCl4)2}.\footnote[frame]{Zdroj: \href{https://commons.wikimedia.org/wiki/File:Cadmium(I)-tetrachloroaluminate-xtal-1987-unit-cell-CM-3D-ellipsoids.png}{Ben Mills/Commons}}}
	\end{figure}
	\vfill
}

\subsection{Rtuť}
\frame{
	\frametitle{}
	\vfill
	\begin{itemize}
		\item Chemicky se mírně liší od zinku a kadmia.
		\item S kyselinami nereaguje, s výjimkou oxidujících -- dusičnou a koncentrovanou kyselinu sírovou.
		\item Kovy rozpouští za vzniku \textit{amalgámů}.\footnote[frame]{\href{https://www.youtube.com/watch?v=IrdYueB9pY4}{Aluminum and Mercury}}
		\item Rtuť vytváří sloučeniny v oxidačních stavech I a II.
		\item V oxidačním stavu I vytváří dikation \ce{Hg$_2^{2+}$} s vazbou \ce{Hg-Hg}.
		\item Působením silných ligandů, jako jsou např. kyanidy, dochází k disproporcionaci na \ce{Hg^{2+}} a kovovou rtuť.
		\item Nejběžnějším oxidačním stavem je II.
		\item Halogenidy rtuťnaté jsou lineární. Vytvářejí tetrahalogenidové komplexy, které mají tetraedrickou symetrii:
		\item \ce{HgCl2 + 2 KCl -> K2[HgCl4]}
	\end{itemize}
	\vfill
}

\frame{
	\frametitle{}
	\vfill
	\textbf{Polykationty rtuti}
	\begin{itemize}
		\item Vazbu mezi atomy rtuti v kationtu \ce{Hg$_2^{2+}$} můžeme popsat překryvem orbitalů 6s.\footnote[frame]{\href{https://doi.org/10.1007/BF02903655}{On formation of polyatomic mercury cations}}
		\item Známe i větší polykationty rtuti, reakcí v tavenině můžeme získat lineární kation \ce{Hg$_3^{2+}$}:
		\item \ce{2 Hg + HgCl2 + 2 AlCl3 -> Hg3[AlCl4]2}
		\item Oxidací rtuti pomocí \ce{AsF5} získáme kation \ce{Hg$_4^{2+}$}:\footnote[frame]{\href{https://doi.org/10.1021/ic00151a015}{Preparation and crystal structure of tetramercury bis(hexafluoroarsenate) \ce{Hg4(AsF6)2}}}
		\item \ce{4 Hg + 3 AsF5 ->[-196 $^\circ$C, SO2] Hg4[AsF6]2 + AsF3}
	\end{itemize}

	\begin{center}
	\begin{tabular}{|c|c|}
		\hline
		\textbf{Kation} & \textbf{Délka vazby \ce{Hg-Hg} [pm]} \\\hline
		\ce{Hg$_2^{2+}$} & 249--256 \\\hline
		\ce{Hg$_3^{2+}$} & 251--255 \\\hline
		\ce{Hg$_4^{2+}$} & 259--267 \\\hline
	\end{tabular}
	\end{center}
	\vfill
}

\frame{
	\frametitle{}
	\vfill
	\begin{figure}
		\adjincludegraphics[height=.75\textheight]{img/35412-ICSD.png}
		\caption*{Struktura \ce{Hg4(AsF6)2}}
	\end{figure}
	\vfill
}

\frame{
	\frametitle{}
	\vfill
	\textbf{Organokovové sloučeniny rtuti}
	\begin{itemize}
		\item Známe velké množství organokovových sloučenin rtuti.
		\item Většina má stechiometrii RHgX nebo \ce{R2Hg}, sloučeniny jsou zpravidla tepelně a fotochemicky nestabilní.
		\item Připravují se reakcí sodného amalgámu s organickými halogenidy:
		\item \ce{2 Hg + 2 RX -> R2Hg + HgX2}
		\item \ce{HgX2 + 2 Na -> Hg + 2 NaX}
		\item Nebo reakcí Grignardova činidla s \ce{HgCl2}:
		\item \ce{RMgX + HgCl2 ->[THF] RHgCl+ MgXCl}
		\item \ce{RMgX + RHgCl ->[THF] R2Hg + MgXCl}
		\item RHgX jsou krystalické látky, \ce{R2Hg} jsou vysoce toxické kapaliny nebo nízkotající látky.
		\item Jsou tvořeny lineárními jednotkami \ce{R-Hg-R} nebo \ce{R-Hg-X}.
		\item Rtuť si v těchto sloučeninách zpravidla zachovává koordinační číslo 2.
	\end{itemize}
	\vfill
}

\frame{
	\frametitle{}
	\vfill
	\begin{figure}
		\adjincludegraphics[height=.75\textheight]{img/o-fenylhydrargium.png}
		\caption*{Trimer \textit{o}-fenylenhydrargyria}
	\end{figure}
	\vfill
}

\frame{
	\frametitle{}
	\vfill
	\begin{columns}
		\begin{column}{.7\textwidth}
			\textbf{Dimethylrtuť}
			\begin{itemize}
				\item Dimethylrtuť, \ce{(CH3)2Hg}, je silně toxická, těkavá kapalina.
				\item Vysoká toxicita je dána i schopností prostupovat kůží, smrtelná dávka pro člověka je 5 mg.kg$^{-1}$.\footnote[frame]{\href{https://cen.acs.org/safety/lab-safety/25-years-Karen-Wetterhahn-died-dimethylmercury-poisoning/100/i21}{25 years after Karen Wetterhahn died of dimethylmercury poisoning, her influence persists}}
				\item Molekula je lineární.
				\item Připravuje se reakcí sodného amalgámu s methyljodidem:
				\item \ce{Hg + 2 Na + 2 CH3I -> Hg(CH3)2 + 2 NaI}
				\item Využívá se jako primární standard v $^{199}$Hg a $^{201}$Hg NMR.\footnote[frame]{\href{http://chem.ch.huji.ac.il/nmr/techniques/1d/row6/hg.html}{(Hg) Mercury NMR}}
			\end{itemize}
		\end{column}
		\begin{column}{.35\textwidth}
			\begin{figure}
				\adjincludegraphics[width=\textwidth]{img/Dimethyl-mercury-3D-vdW.png}
				\caption*{Struktura dimethylrtuti.\footnote[frame]{Zdroj: \href{https://commons.wikimedia.org/wiki/File:Dimethyl-mercury-3D-vdW.png}{Benjah-bmm27/Commons}}}
			\end{figure}
		\end{column}
	\end{columns}
	\vfill
}

\section{Biologie}
\frame{
	\frametitle{}
	\vfill
	\begin{itemize}
		\item Zinek patří mezi nejdůležitější kovy pro rostliny i živočichy.
		\item Lidské tělo obsahuje asi 2--4 g zinku, většinu ve formě enzymů.
		\item Zinek je Lewisovská kyselina, proto je z katalytického hlediska velice zajímavý.
		\item Také je velice flexibilní z hlediska koordinační geometrie, proto umožňuje rychlou změnu konformace enzymu.
		\item Nedostatek zinku se projevuje mnoha symptomy:\footnote[frame]{\href{https://ods.od.nih.gov/factsheets/Zinc-HealthProfessional/}{Zinc}}
		\begin{itemize}
			\item lámavostí vlasů a nehtů
			\item suchou a popraskanou kůží
			\item zpomalením růstu u dětí
			\item šeroslepostí
			\item nechutenstvím
		\end{itemize}
		\item Na rozdíl od zinku jsou sloučeniny kadmia a rtuti velmi toxické.
	\end{itemize}
	\vfill
}

\input{../Last}

\end{document}